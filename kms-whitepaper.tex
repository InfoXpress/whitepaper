\documentclass[notitlepage,longbibliography]{revtex4-1}

% Project name
\newcommand{\kms}{SmartKMS}

\usepackage{graphicx}
\usepackage{amsmath}
\usepackage[margin=5pt]{subfig}
\usepackage[usenames]{color}
\usepackage{librebaskerville}
% \usepackage{xspace}
\definecolor{darkgreen}{rgb}{0.00,0.50,0.25}
\definecolor{darkblue}{rgb}{0.00,0.00,0.67}
\newcommand{\figref}[1]{Fig.~\ref{#1}}
\usepackage[breaklinks,pdftitle={SmartKMS: blockchain-based encryption as a service}, pdfauthor={Michael Egorov},colorlinks,urlcolor=blue,citecolor=darkgreen,linkcolor=darkblue]{hyperref}
\usepackage[usenames]{color}
\graphicspath{{pdf/}}

\begin{document}

\title{\kms~--- blockchain-based key management as a service}

\author{M. Egorov}
\email{michael@nucypher.com}
\author{M. Wilkison}
\email{maclane@nucypher.com}
\affiliation{NuCypher}

% TODO: include David Nunez when he agrees

\begin{abstract}
    \kms~is a global, world-sized Key Management System.
    It provides encryption and access management functionality as a service, performed by the decentralized network,
    with the help of proxy re-encryption~\cite{wiki:pre}.
    Importantly, it doesn't rely on trusting a service provider, unlike centralized encryption-as-a-service solutions.
    \kms~enables sharing of sensitive data for both decentralized and centralized applications,
    giving a promise to revolutionize healthcare, secure identity data and enable creation of decentralized pay-for-content applications.
\end{abstract}

\date{\today}
\maketitle

\section{Introduction}

The \kms~is a blockchain-based encryption and access control management as a service.

(Overview of encryption: symmetric encryption, public key encryption, encryption at rest, TLS).

(Overview of key management systems: what they do, why, KMS as a service - CloudHSM, truevault; self-run: hashicorp).

(Brief statement that we enable KMSaas with combination of blockchain and PRE)

\bibliography{kms-whitepaper}

\end{document}
